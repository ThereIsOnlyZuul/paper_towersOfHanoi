\documentclass[12pt]{article}

\usepackage[margin=1in]{geometry}
\usepackage{booktabs}
\usepackage{tikz}

\usepackage{fontspec}
\setmainfont{Source Serif Pro}
\setsansfont{Source Sans Pro}
\setmonofont{Source Code Pro}

\title{The Towers of Hanoi: A fun math game in $2^n-1$ easy steps}
\author{Derek Klinge}
\date{}

%% Custom Commands
% Shorthand
\newcommand{\ToH}{\textbf{Towers of Hanoi}}

\begin{document}

\maketitle

\section{Introduction}

The {\ToH} is a classic puzzle that has intersting mathematical properties. The
puzzle was first introduced in 1883 by Eduoard Lucas\cite{palmer1996exploring}.
In this paper, we will primarily consider the classic version of the problem, though many
interesting variations exist and will be discussed at the end. We will start by
limiting ourselves to a problem with three disks, and increase the number of
disks once we sufficiently understand the problem.

The basic version of {\ToH} begins with three towers. All of the disks are
arranged from larges to smallest on the first tower. Our goal is to move the
complete stack of disks from the firs tower to the third tower, but we must do
so following specific rules. 

The first rule is that only one disk may be moved
at one time. This means that you cannot grab several disks and move them
together in a single move, you must move each disk seperately. That rule is easy
enough to follow.

The second rule is slightly more complex. When you move a disk, you cannot ever
place a large disk on top of a smaller one. This is a bit hard to explain
without looking at some examples, so we will do that now.


\bibliographystyle{plain}
\bibliography{towers}  

\end{document}
